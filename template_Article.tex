\documentclass[]{article}
\usepackage{graphicx}
\graphicspath{ {images/} }
\usepackage{fancyhdr}
\pagestyle{fancy}
\usepackage[spanish]{babel}
\usepackage{booktabs}
\usepackage{caption}

%opening

\lhead{\begin{picture}(0,0) \put(0,0){\includegraphics[width=20mm]{./logou.png}} \end{picture}}

\title{Tarea 1 de IN1017 Matemáticas Discretas Combinatoria y Grafos}
\author{Karin Momberg}

\begin{document}

\begin{titlepage}
		\centering
		\includegraphics[width=0.3\textwidth]{logo.png}\par\vspace{1cm}
		{\scshape\LARGE Universidad de Aysén \par}
		\vspace{1cm}
		{\scshape\Large IN1017 Matemáticas Discretas Combinatoria y Grafos\par}
		\vspace{1.5cm}
		{\huge\bfseries Tarea 1\par}
		\vspace{2cm}
		{\Large\itshape Karin Momberg\par}
		
		\vfill
		% fin de pag
		{\large \Today\par}
	\end{titlepage}

%\maketitle
Considerando la función exponencial y utilizando el software Octave se calcularon las siguientes funciones para los valores de x en el intervalo (0;5), utilizando un $\Delta x = 0,25$: \\
$e^{x}$ \hspace{5mm} La función prederminada de Octave “Valor real”. \\
$p(x)$ \hspace{2mm} Expansión de Taylor de orden 4. \\
$e^{x}_{4}$ \hspace{5mm} Utilizando la aproximación de e, de orden 4, elevar a x. \\
$e_{err1}$ \hspace{1mm} Diferencia entre ambas funciones ($e^{x} - p(x)$). \\
$e_{err2}$ \hspace{1mm} Diferencia entre ambas funciones ($e^{x} - e^{x}_{4}$). \\
$R_{L}$ \hspace{3mm} Resto de Lagrange, que aproxima el error de truncación del polinomio $p(x)$. \\\\

\begin{flushleft}
	Resultando en la siguiente gráfica: 
\end{flushleft}
\begin{center}
	\includegraphics[width=\textwidth]{xd} \\ 
\end{center}
\begin{flushleft}
	Figura 1. ex corresponde a $e^{x}$ (función predeterminada), p(x) corresponde a la expansión de Taylor de orden 4. ex4 corresponde a la aproximación de e de orden 4 elevada a x. err1 corresponde a la diferencia entre la $e^{x} - p(x)$. RL corresponde al error de Lagrange y err2 corresponde a la diferencia entre $e^{x} - e^{x}_{4}$. 
\end{flushleft}
	a) Indique el error absoluto de $e$, al aproximarlo con un polinomio de orden 4. 

\begin{flushleft}
	 Se define el error absoluto como:
\end{flushleft}

\begin{center}
	$E_{A} (x^{*}) = |x-x^{*}|$
\end{center} 

\begin{flushleft}
			Considerando $x = e_{4}$ como la aproximación y $x^{*} = e$ como "valor ideal",  el error absoluto corresponde a:
\end{flushleft}

\begin{center} $E_{A} (e) = |e_{4}-e|$ \\
\end{center}

\begin{flushleft}
	Se obtiene la aproximación de $e$ con un polinomio de orden 4 ($e_{4}$):
\end{flushleft}

\begin{center}
	$e_{4}$ = $\displaystyle\sum_{j=0}^{4} \displaystyle\frac{1^{j}}{j!}$ = $1 + \displaystyle\frac1{1!} + \displaystyle\frac1{2!} + \displaystyle\frac1{3!} + \displaystyle\frac1{4!}$ \\ 
	= $1 + 1 + \displaystyle\frac1{2} + \displaystyle\frac1{6} + \displaystyle\frac1{24} = 2.7083$	
\end{center}

\begin{flushleft}
Con $e_{4} = 2,7083$ y $e = 2,7183$ (Valor determinado por Octave). 
\end{flushleft}

\begin{center} $E_{A} (e) = |2,7083-2,7183| = 0,0100 $	
\end{center}

\begin{flushleft}
	Por lo tanto el  error absoluto corresponde a $0,0100$.\\
\end{flushleft}


\begin{flushleft}
	b) Al aproximar $e$, por $e_{4}$, se tiene que $e = e_{4} + \Delta e$. Al aproximar $e^{5}$ utilizando como base $e_{4}$, resulta $e^{5} = e_{4}^5 + E_{T}$. Demuestre entonces que:
\end{flushleft}

\begin{center} $E_{T} = 4 \cdot e_{4}^4 \cdot \Delta e + k_{1} \cdot \Delta {e}^2 + k_{2} \cdot \Delta e^{3} + ...$ 
\end{center} 

\begin{flushleft}
		Donde $K_{i}$, son constantes. \\
\end{flushleft}
			
\begin{flushleft}
	Si aproximamos $e^{5}$ con $e$ = $e_{4} + \Delta{e}$, obtenemos $e^{5} = (e_{4} + \Delta{e})^{5}$. Para resolver se utiliza, en este caso, el teorema del binomio. \\
Teorema del binomio en su forma general: 
\end{flushleft}
\begin{center}
	$(x+y)^n = \displaystyle\sum_{k=0}^{n} {{n}\choose{k}} \cdot x^{n-k}y^k $

\end{center}

\begin{flushleft}
	En este caso sería:
\end{flushleft}

\begin{center}
	$(e_{4} + \Delta{e})^{5} = \displaystyle\sum_{k=0}^{5} {{5}\choose{k}} \cdot e_{4}^{5-k}\cdot\Delta e^k $
\end{center}

\begin{flushleft}
	Al resolver obtenemos la siguiente expresión: \\
\end{flushleft}
	
\begin{center}
$e^{5} = (e_{4} + \Delta{e})^{5} = e_{4}^{5} + 5\cdot{e^{4}_{4}}\cdot\Delta{e} + 10\cdot{e^{3}_{4}}\cdot\Delta{e^{2}} + 10\cdot{e^{2}_{4}}\cdot\Delta{e^{3}} +  5\cdot{e_{4}}\cdot\Delta{e^{4}} +  \Delta{e^{5}}$
\end{center}
	
\begin{flushleft}
	Teniendo en cuenta que $e^{5} = e^{5}_{4} + E_{T} $, Igualamos al resultado anterior: \\
\end{flushleft}
	
\begin{center}
$e^{5}_{4} + E_{T} = e_{4}^{5} + 5\cdot{e^{4}_{4}}\cdot\Delta{e} + 10\cdot{e^{3}_{4}}\cdot\Delta{e^{2}} + 10\cdot{e^{2}_{4}}\cdot\Delta{e^{3}} + 5\cdot{e_{4}}\cdot\Delta{e^{4}} + \Delta{e^{5}}$
\end{center}

\begin{flushleft}
	Se cancelan términos iguales y se reemplazan las constantes por $K_{i}$ \\ ($k_{1} = 10\cdot{e^{3}_{4}}$ y $k_{2} = 10\cdot{e^{2}_{4}}$) resultando:
\end{flushleft}

\begin{center}
	$E_{T}$ = $5\cdot{e^{4}_{4}}\cdot\Delta{e}$	+ $k_{1}\cdot\Delta{e^{2}}$ + $k_{2}\cdot\Delta{e^{3}}$ + ...\\
\end{center}

\begin{flushleft}
	Es fácil ver que hay un error en el primer término de la expresión presente en el enunciado, cuyo valor debería ser "5", en vez de "4".
Por lo tanto queda demostrado que $E_{T} \neq 4 \cdot e_{4}^4 \cdot \Delta e + k_{1} \cdot \Delta {e}^2 + k_{2} \cdot \Delta e^{3} + ...$ . \\
\end{flushleft}
	

\begin{flushleft}
	c) Si se desprecian los valores de orden superior ($\Delta e^{2}, \Delta e^{3}, ...$), Demuestre que: \\
\end{flushleft}

\begin{center} $E_{T} \approx (x-1) \cdot e_{4}^{x-1}\cdot \Delta e$
\end{center}

	
\begin{flushleft}
	En primer lugar, se puede utilizar la expresión anterior, despreciando los valores de orden superior:

\end{flushleft}
\begin{center}
	$E_{T} = 5\cdot{e^{4}_{4}}\cdot\Delta{e}$ , con $x = 5$.
\end{center}

\begin{flushleft}
	Si reescribimos la expresión $E_{T}$ de forma generalizada, obtenemos:
\end{flushleft}

\begin{center}
	$E_{T} = x\cdot{e_{4}^{x-1}}\cdot\Delta{e}$ 
\end{center}

\begin{flushleft}
	Al igualar el $E_{T}$ obtenido de la demostración anterior al presentado en el enunciado, es fácil ver que $x\cdot{e_{4}^{x-1}}\cdot\Delta{e} \neq (x-1) \cdot e_{4}^{x-1}\cdot \Delta e$ por lo que queda demostrado que $E_{T} \neq (x-1) \cdot e_{4}^{x-1}\cdot \Delta e$, sin embargo, cabe destacar que se cumpliría ssi el caso anterior estuviera correcto.\\
\end{flushleft}

	
\begin{flushleft}
	¿Es válida para todo x, o existe alguna restricción al respecto? \\
\end{flushleft}

\begin{flushleft}
	El teorema del binomio utilizado sólo sirve en el caso de valores de x positivos enteros. Sin embargo, Newton generalizó la fórmula para exponentes reales, considerando una serie infinita:
\end{flushleft}

\begin{center}
	$(x + y)^{r}$ = $\displaystyle\sum_{i=0}^{\infty} {r \choose i} x^{r-i}y^{i}$
\end{center}
En términos del enunciado queda de la siguiente manera:

\begin{center}
	$(e_{4} + \Delta{e})^{r} = \displaystyle\sum_{i=0}^{\infty} {r \choose i} e_{4}^{r-i}\cdot\Delta{e}^{i}$
\end{center}
Al desarrollar el binomio de Newton generalizado obtenemos:

\begin{center}
	$(e_{4} + \cdot\Delta{e})^{r} = \displaystyle\sum_{i=0}^{\infty} {r \choose i} e_{4}^{r-i}\cdot\Delta{e}^{i} = \displaystyle{r \choose 0}e_{4}^{r}\cdot\Delta{e}^{0} +  \displaystyle{r \choose 1}e_{4}^{r-1}\cdot\Delta{e}^{1} +  \displaystyle{r \choose 2}e_{4}^{r-2}\cdot\Delta{e}^{2}$ + ...
\end{center}
Al despreciar los valores de orden superior ($\Delta e^{2}, \Delta e^{3}, ...$), resulta la siguiente expresión:

\begin{center}
	$(e_{4} + \Delta{e})^{r} = \displaystyle\sum_{i=0}^{\infty} {r \choose i} e_{4}^{r-i}\cdot\Delta{e}^{i} = \displaystyle{r \choose 0}e_{4}^{r} +  \displaystyle{r \choose 1}e_{4}^{r-1}\cdot\Delta{e}$
\end{center}

\begin{flushleft}
	Luego, se resuelven las combinatorias:
	
\end{flushleft}
\begin{center}
	$(e_{4} + \Delta{e})^{r}$ = $\displaystyle\frac{r!}{r!}e_{4}^{r}$ +  $\displaystyle\frac{r(r-1)!}{(r-1)!}e_{4}^{r-1}\cdot\Delta{e}$
\end{center}

\begin{flushleft}
	Como resultado tenemos que: 
\end{flushleft}

\begin{center}
	$(e_{4} + \Delta{e})^{r}$ = $e_{4}^{r}$ +  $r\cdot{e_{4}^{r-1}}\cdot\Delta{e}$
\end{center}

 
\begin{flushleft}
	 y su error de truncación correspondería a $E_{T} = r\cdot{e_{4}^{r-1}}\cdot\Delta{e}$, muy parecido al caso de los exponentes positivos enteros, a diferencia de que en exponentes reales se considera una serie infinita. Se puede apreciar que nuevamente no se cumple el enunciado. \\

\end{flushleft}
Demuestre además que el error relativo es posible aproximarlo: 
\begin{center} $E_{R} \approx (x-1) \cdot e_{4}^{-1}\cdot \Delta e$ \end{center} 
\begin{flushleft}
	Se define el error relativo como:
\end{flushleft}

\begin{center} $E_{R} (x^{*}) = \displaystyle\frac{|x-x^{*}|}{|x|} $ \\
\end{center}

\begin{flushleft}	
Al reemplazar obtenemos lo siguiente:
\end{flushleft}

\begin{center}
	$E_{R}(x^{*}) \approx\displaystyle\frac{x\cdot{e_{4}^{x-1}}\cdot\Delta{e}}{e^{x}_{4}}$
\end{center}

\begin{flushleft}
	Se resuelve:
\end{flushleft}

\begin{center}
	$E_{R}(x^{*}) \approx\displaystyle{x\cdot{e_{4}^{-1}}\cdot\Delta{e}}$
\end{center}

\begin{flushleft}
	Dado el resultado se pudo demostrar que:
\end{flushleft}
\begin{center}
	 $\displaystyle{x\cdot{e_{4}^{-1}}\Delta{e}} \neq (x-1) \cdot e_{4}^{-1}\cdot \Delta e$
\end{center} 
	

\begin{flushleft}
		d) Compare la relación obtenida en letra c) con los valores de $e_{rr2}$. \\
\end{flushleft}



\begin{tabular}{|r|r|}
	
		\toprule
		\multicolumn{1}{|l|}{$e_{T}$} & \multicolumn{1}{l|}{$e_{rr2}$} \\
		\midrule
		0,00000 & 0,00000 \\
		\midrule
		0,00118 & 0,00118 \\
		\midrule
		0,00304 & 0,00302 \\
		\midrule
		0,00585 & 0,00581 \\
		\midrule
		0,01000 & 0,00995 \\
		\midrule
		0,01604 & 0,01596 \\
		\midrule
		0,02469 & 0,02458 \\
		\midrule
		0,03695 & 0,03681 \\
		\midrule
		0,05417 & 0,05399 \\
		\midrule
		0,07817 & 0,07795 \\
		\midrule
		0,11143 & 0,11116 \\
		\midrule
		0,15724 & 0,15693 \\
		\midrule
		0,22005 & 0,21972 \\
		\midrule
		0,30582 & 0,30550 \\
		\midrule
		0,42250 & 0,42225 \\
		\midrule
		0,58071 & 0,58065 \\
		\midrule
		0,79463 & 0,79491 \\
		\midrule
		1,08311 & 1,08398 \\
		\midrule
		1,47119 & 1,47305 \\
		\midrule
		1,99217 & 1,99561 \\
		\midrule
		2,69016 & 2,69604 \\
		\bottomrule
\end{tabular}
\label{tab:comparar}
\begin{flushleft}
	Tabla 1: Comparación $e_{T}$ y $e_{rr2}$
\end{flushleft}



\begin{flushleft}
	e) Discuta:
\end{flushleft}

\begin{itemize}
	\item De las dos alternativas propuestas, ¿Cuál es la mejor forma de obtener una buena aproximación de $e^{x}$?
\end{itemize}

\begin{flushleft}
	La mejor forma de obtener una buena aproximación es con $e_{rr2}$, pese a ser muy similar a la del $e_{T}$, el error es más pequeño.
\end{flushleft}
\begin{itemize}
	\item Suponga que usted propone obtener una buen aproximación de e, digamos con n iteraciones i.e. $e^{x} = e_{n}^{x} + E_{T}$ (en letra b) se usó n = 4). Qué valor de n deberíamos escoger para que $E_{T} \approx (x-1) \cdot e_{4}^{x-1} \cdot \Delta e < 1$ para $x = 10$. \\
\end{itemize}

 	
 	Se debería escoger $n = 8$ ya que el $e_{T} = 0,22306$, mientras que con $n = 7 $
 	se obtiene $e_{T} = 2,0316$, pasándose de 1.
 	\begin{flushleft}
 		Para obtener dichos valores se utilizó el siguiente código en Octave:
 	\end{flushleft}
\begin{center}
	 	\includegraphics{../b} \\
\end{center}
 	
\section*{Anexo}
\begin{flushleft}
	Codigo Octave (funciones):
\end{flushleft}
\begin{center}
	\includegraphics{../1}
\includegraphics{../2}
\end{center}

\begin{flushleft}
	Códigos Octave y LaTeX: https://github.com/Kheirily/MatematicasDiscretas
\end{flushleft}

\end{document}
