\documentclass[]{article}
\usepackage{graphicx}
\graphicspath{ {images/} }

%opening
\title{Tarea 1 de IN1017 Matemáticas Discretas Combinatoria y Grafos}
\author{Karin Momberg}

\begin{document}

\maketitle
Considerando la función exponencial y utilizando el software Octave se calcularon las siguientes funciones para los valores de x en el intervalo (0;5), utilizando un $\Delta x = 0,25$: \\
$e^{x}$ \hspace{5mm} La función prederminada de Octave “Valor real”. \\
$p(x)$ \hspace{2mm} Expansión de Taylor de orden 4. \\
$e^{x}_{4}$ \hspace{5mm} Utilizando la aproximación de e, de orden 4, elevar a x. \\
$e_{err1}$ \hspace{1mm} Diferencia entre ambas funciones ($e^{x} - p(x)$). \\
$e_{err2}$ \hspace{1mm} Diferencia entre ambas funciones ($e^{x} - e^{x}_{4}$). \\
$R_{L}$ \hspace{3mm} Resto de Lagrange, que aproxima el error de truncación del polinomio $p(x)$. \\\\
Resultando en la siguiente gráfica:

\includegraphics[width=\textwidth]{xd} \\\\


a) Indique el error absoluto de $e$, al aproximarlo con un polinomio de orden 4. \\\\



Dado $E_{A} (x^{*}) = |x-x^{*}|$ y Considerando $x = e_{4}$ como la aproximación y $x^{*} = e$ como "valor ideal": Se tiene que el error absoluto de $e$ corresponde a: \\\\
\begin{center} $E_{A} (e) = |e_{4}-e|$ \\
	= fhfghfghdgdfg
\end{center}

Al Calcular el polinomio de orden 4 de e, se obtiene:
\begin{center}
	contenido[inserte polinomio]s...
\end{center}

...

b) Al aproximar $e$, por $e_{4}$, se tiene que $e = e_{4} + \Delta e$. Al aproximar $e^{5}$ utilizando como base $e_{4}$, resulta $e^{5} = e_{4}^5 + E_{T}$. Demuestre entonces que:
\\\\
\begin{center} $E_{T} = 4 \cdot e_{4}^4 \cdot \Delta e + k_{1} \cdot \Delta {e}^2 + k_{2} \cdot \Delta e^{3} + ...$ \end{center} 

Donde $K_{i}$, son constantes. \\\\

c) Si se desprecian los valores de orden superior ($\Delta e^{2}, \Delta e^{3}, ...$), Demuestre que: \\\\
\begin{center} $E_{T} \approx (x-1) \cdot e_{4}^{x-1}\cdot \Delta e$ \end{center}

¿Es válida para todo x, o existe alguna restricción al respecto?

Demuestre además que el error relativo es posible aproximarlo:
\begin{center} $E_{R} \approx (x-1) \cdot e_{4}^{-1}\cdot \Delta e$ \end{center}
d) Compare la relación obtenida en letra c) con los valores de $e_{rr2}$. \\

e) Discuta:

\begin{itemize}
	\item De las dos alternativas propuestas, ¿Cuál es la mejor forma de obtener una buena aproximación de $e^{x}$?

	\item Suponga que usted propone obtener una buen aproximación de e, digamos con n iteraciones i.e. $e^{x} = e_{n}^{x} + E_{T}$ (en letra b) se usó n = 4). Qué valor de n deberíamos escoger para que $E_{T} \approx (x-1) \cdot e_{4}^{x-1} \cdot \Delta e < 1$ para $x = 10$. \\
	
\end{itemize}


\section{}

\end{document}
